\chapter{绪论}
\label{cha:xulun}

\section{细粒度图像分类的研究背景与意义}
\label{sec:beijing}
本世纪是一个信息大爆炸的时代,伴随着计算机和网络技术的快速发展,全球互联网进入中速发展期,其用户规模实现平稳增长,现实生活中所接收到的信息也越来越多,尤其是多媒体信息。并且根据2017年互联网中心发布的一份关于互联网发展的权威报告~\cite{jin}截至2016年12月,中国网站数量为482万个,年增长14.1\%。而作为目前媒体信息中出现的最多的图像信息,由于其包含大量信息,更让人们关注。图片信息之庞大,面对这种复杂的局面,仅仅依靠人工进行图像信息筛选十分困难,所以要借用计算机技术来帮助人们解决问题是非常必要的。人类可以使计算机了解图像,这种方法就是图像分类,这就说明对图像分类这个领域的探索具有巨大的实际价值。并且近几年计算机视觉、人工智能等领域逐渐进入人们的视野,这使图像分类技术有了新的进展,所获得的成果也体现在生活中的各个方面。
  
图像分类的对象是从确定的类别标签的集合中预测给出图像所属于的种类。在计算机视觉的研究过程里,图像分类主要分为粗糙分类和细分类两种形式。所谓的粗糙分类,就是传统图像分类,用来区分物体属于哪一个物种。细分类就是细粒度图像分类,用来区分对象属于物种中的哪一类。经过神经科学的逐渐发展和进步,研究人员认识到人类识别物体是借助物体具有区分度的局部特征,并且把它们组合起来实现的。这说明, 学习一个高层概念需要建立结构性的模型。但是,传统的图片分类结果包括了大部分的无用的信息,这种情况下抑制了计算机算法经过一定次数的训练数据学习得到结构性的模型。此时图像分类中的一个更加深入的领域——细粒度图像分类(Fine-Grained Image Classification)走进我们的视线。在细粒度图像分类领域,待分类图像的所属类别的大体特征很相近,因此研究者可以从全部图像数据集的共同拥有的特征信息中学习到更好的层次结构模型。
\section{细粒度图像分类的研究现状}
\label{sec:xianzhuang}
传统图像分类的方法依赖于大量的人工标注,结果不理想而且过程还很繁琐,准确度大部分情况下依赖标注者的主观意识。并且,人工标注信息耗费的人力时间都很多,所以目前新兴的方法大部分偏向弱监督,无需人工标注只需要标签就可以自动训练,并且有一些算法的准确度可以达到百分之八十多。

比较经典的算法是,2011年Wah~\cite{WelinderEtal2010}等人在发布CUB-200-2011数据库的技术报告中所做的基准测试,其结果仅为10.3\%。算法过程是,使用一幅原始的、未经过任何标注的测试图片,利用训练过程得到的模型进行局部区域位置的定位;之后提取特征,经过词袋(Bag-of-Words, BoW)~\cite{harris1954distributional}模型进行特征编码后,输入到分类器完成分类。但如果在测试时给定了标注框和局部区域位置这些标注信息的话,利用同样的方法,得到的基准测试结果为17.3\%。之后,Berg T~\cite{berg2013poof}等人提出一种方法基于部位的一对一特征,POOF可以自主地从一组特定区域的具有固定位置和类别标注信息的图片集中学习不同的具有高度区别的中等特征。这种方法精确度最高可以达到73.3\%。

但是,传统方法的准确度仍然不理想,近年来兴起的深度学习为我们在图像分类领域指明了新的方向。总的来说,深度学习对数据特征的无监督学习,使它具有深层复杂的网络结构布局。现在深度学习技术在图像分类领域的应用己经展现出非常明显的优势。

特征表示是细粒度识别的一个关键问题,卷积神经网络(CNN)~\cite{sharif2014cnn}被广泛应用于特征提取。然而,细粒度表示存在两个挑战。首先,传统的CNN表示需要固定大小的矩形作为输入,这不可避免地包括背景信息。然而,背景不太可能对细粒度的识别起任何重要作用,因为所有的子类都有相似的背景。针对这个问题Xiaopeng Zhang~\cite{zhang2016picking}等人提出了一种基于深度滤波的细粒度图像分类方法,将CNN的深层过滤反应作为局部描述符,并通过SWFV-CNN对其进行编码。第一步的目标是挑选深度过滤器,以显著和一致地响应特定的模式。第二步是通过fisher向量的空间加权组合来选择CNN滤波器。实验结果表明,SWFV的性能优于传统的CNN,与传统的CNN相辅相成,进一步提高了性能。

现有的基于CNN的细粒度分类方法并不侧重于对局部语义的检测和利用。所以对于鸟类幼崽的分类来说识别头部的效果总是比身体差,因为头部的尺寸很小。传统的part-based CNN 方法对于部分网络之间的卷积过滤器的共享有困难。来自俄罗斯大学的Han Zhang~\cite{zhang2016spda}等人关注到富有语义的局部,认识到大多数卷积神经网络缺乏模型化对象语义部分的中层。故提出了一种新的CNN架构, 构建了一个End-to-End的网络,将语义部分检测与抽象相结合进行细粒度分类。该网络有两个子网络,一个用于检测一个用于识别,该检测子网络具有一种新颖的自顶向下的生成小语义部分候选检测方法。分类子网络引入了一种新的部件层,从检测子网络检测到的部分提取特征,并结合它们进行识别。这种部分特定的学习为更深入地了解细粒度类别打开了门,不仅仅是识别类标签。

目前,大部分方法都侧重于深度学习,这使细粒度图像分类的准确度有了很大的提高。实现的方法有很多这里就不一一赘述了。
\section{本文的研究内容及组织结构}
\label{sec:jiegou} 
本文共分五章,具体内容介绍如下:

第一章为绪论,对细粒度图像分类领域的研究背景与研究意义进行综述,简单阐述了细粒度图像分类方法的国内外研究现状,为后文各个章节内容的介绍作铺垫。

第二章主要介绍细粒度图像分类技术的基础知识,基本概念,细粒度图像分类方法的概述及时下常用的几种用于细粒度图像分类的数据库。

第三章主要介绍细粒度图像分类问题的传统解决方法,介绍三种提取特征的方法,词袋模型和分类器。

第四章主要介绍用第三章所提到的三种方法进行实验和实验分析。

第五章主要对论文进行总结和未来的展望。
