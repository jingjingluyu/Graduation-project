\chapter{细粒度图像分类基础}
\label{cha:fenleijichu}

\section{概念}
\label{sec:fenleigainian}
细粒度图像分类作为一个近几年热门的研究方向,越来越受到各方面的关注。细粒度分类属于目标识别的一个子领域,它的目标是区分属于同一基本层次范畴的数百个子类。虽然它与一般对象分类有关,但细粒度分类要求算法在通常仅由细微差别区分的高度相似的对象之间进行区分。这部分问题的一个相同点是不同类别之间的相似度较高,相同类别内的差异性较大,所以细粒度图像分类与传统的图像分类相比难度更大。因此,对象和对应局部的定位和描述成为细粒度识别的关键。目前,精细图像分类的研究工作主要集中在花卉、鸟类、狗类等,而且也得到了很多的研究成果。细粒度图像分类研究的是需要掌握一定的专业常识才能进行对象分类的问题研究。由于近些年计算机发展的很快,又有很多新的方法出现解决了以往存在的种种方面的缺陷。
\section{流程框架}
\label{sec:liucheng}

与普通的图像分类不同,细粒度图像的信噪比较小,只有在细小的局部区域中可以找到包含足够区分度的信息,因此如何高效的找到并且利用这些信息是关键。如今,大多数的分类算法依据的流程如下:首先找到前景对象(鸟)及其局部区域(嘴 、翅膀等),然后分别对这些有用区域进行特征提取,再将所提取的特征进行一定的处理,最后用分类器进行训练和预测。具体流程框架如图~\ref{fig:kuangjia}所示。
\begin{figure}[H] % use float package if you want it here
  \centering
  \includegraphics[height=4cm]{流}
  \caption{流程框架图}
  \label{fig:kuangjia}
\end{figure}
其中,第一步找到需要的前景对象,可以用到人工标注和局部区域位置检测的方式。提取特征时,有很多可以选择的特征算子比如SIFT~\cite{lowe1999object}、HOG~\cite{dalal2005histograms}特征并且现在的大部分方法都把重点放在深度卷积神经网络上。第三步对特征进行适当处理和分类需要用到SVM~\cite{cortes1995support}等分类器。但具体运用哪种需要根据程序来决定。
\section{数据集}
\label{sec:shujuji}
目前大部分的数据集都是针对普通的图像分类设计的,细粒度图像分类数据库的获取相比之就难度非常大了,因为需要很丰富的专业知识才能完成数据的采集和标注。但随着细粒度图像分类领域的研究的越发深入,也有越来越多的细粒度图像数据库出现,这体现了细粒度图像分类在近年来发展之迅速。

目前比较常用的细粒度图像数据库,本文简要说明如下:

CUB-200-2011~\cite{WelinderEtal2010}:来自加利福尼亚理工学院,数据集的内容是鸟类图像,包含200种不同类别,共11788张图像数据。关于标注情况,数据集提供了丰富的人工标注数据,每张图像包含15个局部区域位置,312个二值属性,1个标注框以及语义分割。

Stanford Dogs~\cite{KhoslaYaoJayadevaprakashFeiFei_FGVC2011}:来自斯坦福大学,数据记得内容是狗类图像,包含120种不同类别,共20580张图像数据,但只提供标注框这一个人工标注信息。

Oxford Flowers~\cite{nilsback2008automated}: 来自牛津大学,数据集内容是花类图像,包含两个库分别是17类别和102类别。每个类包含40至258张,共8189张图像。只提供语义分割图像,不包含其他额外的标注信息。

下图~\ref{fig:yangli}为这三种图像数据库的部分展示。但是CUB-200-2011数据库是目前细粒度图像分类领域最经典的,也是最常用的数据库。所以,本文将使用此数据集进行实验。在表\ref{tab:shujuji}中列举了部分数据集。
\begin{figure}[H] % use float package if you want it here
  \centering
  \includegraphics[height=6cm]{类别}
  \caption{三种数据集样例图片}
  \label{fig:yangli}
\end{figure}

\begin{table}[H]
 % \centering
  %\begin{minipage}[t]{0.8\linewidth} % 如果想在表格中使用脚注,minipage是个不错的办法
  \caption[数据集总表]{数据集总表。其中BBox指标注框信息(Bounding Box),Parts指局部区域信息,Attributes指属性标注信息。}
  \label{tab:shujuji}
   \begin{tabularx}{\linewidth}{c|rl}
     \toprule[1.5pt]
      名称 & 类别与图像数量 &  标注信息 \\
     \hline
      CUB-200-2011  & 200类  11788张 & BBox Parts Attributes 语义分割图像\\ \hline
      Stanford Dogs & 120类  20580张 & BBox\\ \hline
      Oxford Flowers & 17/102类  8189张 & 语义分割图像\\ \hline
      Cars & 196类 16185张  & BBox\\ \hline
      FGVC-Aircraft & 120类 10200张 & BBox\\
      \bottomrule[1.5pt]
    \end{tabularx}
  %\end{minipage}
\end{table}














