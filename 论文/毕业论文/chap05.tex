\chapter{总结与展望}
\label{cha:zongjiezhanwang}

\section{论文总结}
\label{sec:zongjie}
近几年,细粒度图像分类问题越来越收到各方面的关注,普通的图像分类问题已经研究到比较成熟的地步。关于细粒度图像分类的研究价值可想而知是非常巨大的,并且还有一定的难度。目前存在的问题,一种情况是针对细粒度图像数据集比较难以采集获取,另一种是现有的很多方法严重依赖人工标注信息,但是这种方法的成本非常巨大,本惬意定程度上制约了算法的实用性。所以最近几年,各种方法都倾向于,仅仅只用类别标签进行分类,而且得到的分类效果也是非常不错的。并且,能否找到具有区分度的局部特征也是限制分类效果的一个重要因素。深度学习的兴起,以及深度卷积神经网络所获得的成功,使人们发现它的优势并用于细粒度图像分类。
 
本文介绍了细粒度图像分类的一些背景和发展现状,以及基本的理论和基础知识。对细粒度图像的分类流程框架和传统方法基本理论进行说明。对于所选用的三种方法进行试验,得到结果后进行数据分析和对比
\section{展望}
\label{sec:zhanwang}
本文只是针对细粒度图像分类的传统方法进行介绍和实验,虽然是比较经典比较基础的方法,但是科学是总在进步的,不断有新的更加高效的算法被提出。本文的实验方法采用了类别标注信息和标注框信息,准确率非常低。近几年的方法基于深度神经网络的细粒度图像分类准确率可以达到80\%以上,并且不需要人工标注信息,所以未来的研究重点必定是这种方式,并且通过对各种先进算法的比较,未来会找到效果更加的方式。







